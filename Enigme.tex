\documentclass{article}

\usepackage[utf8]{inputenc}

\usepackage{amsmath}
\usepackage{amssymb}

\title{Dnigm$\alpha$
~\\
that eventually proves the existence of 
~\\
G$\phi$D
~\\
on 
~\\
$\gamma$hristmaS.
}
\date{
~\\
2019-12-14
}
\author{
~\\
Damiens ROBERT
}
\pagenumbering{arabic}

\begin{document}
  \maketitle
  \section{$\epsilon$nigm$\alpha$}
    \section{$\alpha$ffirmations}
        \begin{itemize}
          \item
            Cantor $\alpha$ tord.
          \item
            Godel a 7ord aussi ! 
          \item
            Les nombres pairs existent Réellement, ou pas ...
          \item
            La réalité physiQue est logiquement ensembliste.
        \end{itemize}
    \section{$\epsilon$nigm$\epsilon$}
    \section{gobble}
      \paragraph{Clues}
	\begin{itemize}
	  \item
            The sign $\equiv$ is a RELATION between a LANGUAGE and a METALANGUAGE.
	  \item
            The part on the LEFT of the sign $\equiv$ is the METALANGUAGE.
	  \item
            The part on the RIGHT of the sign $\equiv$ is the LANGUAGE.
	\end{itemize}
      \paragraph{Elements}
        \begin{itemize}
          \item
            $\dot{\alpha} \equiv \alpha$
            ~\\
            ~\\
            where $\dot{\alpha}$ is a set.
            ~\\
            and
            ~\\
	    where $\alpha$ is a set.
          \item
            $\dot{\epsilon}$ is a set.
          \item
            $\epsilon$ is a set.
          \item
            $substitution$
          \item
            $\dot{n} = I(n)$
            ~\\
            ~\\
            where n is an object of $\epsilon$
          \item
            $\dot{0} = 0$
          \item
	    $\dot{1}$ is an object of $\dot{\epsilon}$
          \item
	    $\omega \equiv \dot{\omega}$
            ~\\
            ~\\
            where $\dot{\omega}$ is a set.
            ~\\
            and
            ~\\
	    where $\alpha$ is a set.
          \item
            $\dot{1} + \dot{\phi} = 1$
          \item
            $1 = \dot{\phi} + \dot{0}$
          \item
            $\dot{\phi}+ \dot{1} = 1$
          \item
            $\dot{0} + MYSTERY = MYSTERY$
        \end{itemize}
  \section{Questions}
    Trouvez la $substitution$ (élément 4) :
  \section{Demonstrate}
  \section{gobble}
    \begin{itemize}
      \item     
	Le théorème du neutre avec les éléments 9 et 11. (Indice: Vous aurez peut-être besoins de la commutativité)
        \begin{equation*}
		CQFD
        \end{equation*}
      \item
        Le théorème de l'Itérateur avec le 10 et le 5.
        \begin{equation*}
		CQFD
        \end{equation*}
      \item
        Le théorème de la commutativité avec le 9 et le 11.
        \begin{equation*}
		CQFD
        \end{equation*}
      \item
        \section{Question}
          Is the element 12 the origin ? (Indice: oui. Use element 7 to prove it)
    \end{itemize}
    \section{gobble}
    \section{Demonstrate the universal théory.}
        ~\\
        ~\\
	Posons $\dot{1} = 1$
        ~\\
        ~\\
        \begin{equation*}
                CQFD
        \end{equation*}
        ~\\
        ~\\
	Cela suffit-il ? (Indice: oui)
        ~\\
        ~\\
	Pouvez-vous le faire ? (Indice: oui)
        ~\\
        ~\\
	Pourriez-vous le faire si ce n'était pas démontrable par l'axiomatique ? (Indice: non)
    \section{Optional question :}
      \paragraph{Est-ce que $\dot{\epsilon} <=> \epsilon$ où $\dot{\epsilon}$ est un ensemble et $\epsilon$ est un ensemble ? (Indice: Pensez à la substitution et au théorème de l'Itérateur)}
      \paragraph{}
      La réponse est oui mais êtes-vous capable de justifier ? Qu'en est-il pour EXPTIME = P (Indice: Le jeu de go peut être résolu en temps linéaire)
      \paragraph{}
      Quel est Réellement la cardinalité de $\omega$ où $\omega$ est l'ensemble universel ? (pour ce faire, utilisez les axiomes de la continuité. Trouvez-les !)
\end{document}

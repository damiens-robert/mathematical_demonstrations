\documentclass{article}

\usepackage[utf8]{inputenc}

\usepackage{amsmath}
\usepackage{amssymb}

\title{$0/0$
~\\
= 
~\\
(0, X$\infty$)
}
\date{
}
\author{
~\\
Damiens ROBERT
}
\pagenumbering{arabic}

\begin{document}
  \maketitle

  \tableofcontents

  \section{The Division as an algorithm}
  \subsubsection{The Metalanguage}
   A pseudocode is a Metalanguage. In a pseudocode, we can define a division as an operation denoted by 
   ~\\
  \begin{verbatim}
    /
  \end{verbatim}
   This operator has 3 different parts which are :
   ~\\
  \begin{verbatim}
    result = MYSTERY
    number = number to divide
    divisor =  number that divides
  \end{verbatim}
  The result of the division in this pseudocode is
  ~\\
  \begin{verbatim}
    result = number / divisor
  \end{verbatim}
  The result is composed of 2 elements which are the quotient and the remainder.
  ~\\
  \begin{verbatim}
    result = (result.quotient, result.remainder)
  \end{verbatim}
  \subsubsection{The Tought}
  The compiler is the Tought. A Tought is what is thinking and it can also be called a Machine. A tought is the operation that translates a Metalanguage in a Language.
  \subsubsection{The Language}
  A programming language (Language) is the translation of a pseudocode (Metalanguage) by human tought (Tought).
  ~\\
  ~\\
  This programming language can than be itself translated into another language, by example, a compiler in order to be executed by a machine.
  ~\\
  \begin{verbatim}
    // Initialisation
    number = number to divide
    divisor =  number that divides
    result =  (quotient = 0, remainder = number)
    // Computing
    while (divisor > result.remainder) {
      result.remainder = number - divisor
      result.quotient = quotient + 1
    }
    // Result
    return result
  \end{verbatim}
  Let's notice that at each step of the iterator, the value of result.remainder and result.quotient varies the same way.
  \section{The remainder of $0/0$}
  The variation of the remainder for a single step is :
  ~\\
  \begin{verbatim}
    result.remainder = number - divisor
  \end{verbatim}
  When doing MYSTERY = 0 / 0, MYSTERY.remainder = 0 - 0 =  0
  ~\\
  When doing the same step n times, we obtain the vector, (0) that we can also write $X(0_n)$
  \section{The quotient of $0/0$}
  The variation of the quotient for a single step is :
  ~\\
  \begin{verbatim}
    result.quotient = quotient + 1
  \end{verbatim}
  When doing MYSTERY = 0 / 0, MYSTERY.quotient = 0 + 1 = +1
  ~\\
  ~\\
  When doing the same step n times, we obtain the vector, $(a_n)$ that we can also write $+(1_n)$
  ~\\
  ~\\
  I will now refer to my text on (+$\infty$) + (-$\infty$) = 0 to justify that $+(1_n)$ resolves to (+$\infty$) by saying that +1 is a positive number.
  ~\\
  ~\\
  Let's now notice that if we decremented the loop instead of incrementing it, only the sign would change, meaning we would then obtain (-$\infty$) as a quotient.
  \section{Conclusion}
  I will now conclude by  saying that there are an infinite possibility for the sign and not only 2 (+ , -). I will demonstrate it in a text talking about symetry.
  ~\\
  ~\\
  \centerline{
  This infinity of possibilities for the signs are denoted by X.
  }
\end{document}
